\chapter{Introduction}
\label{chap:intro}

The autonomous and distributed nature of Internet networks, an intentional consequence of the Internet architecture, can pose difficulty in coordinating networks and their operators to achieve collective action. This observation is visible in a number of the challenges that face the Internet today as it grows ever more important in the social and economic fabric of our world: taking action against spam and malicious behavior, transitioning to a larger address space (IPv6), etc. These are all cases where the costs and benefits of individual action are generally not commensurate. 

At the same time, the Internet has a relatively strong social network and sense of community amongst network operators---arguably a historical artifact \cite{Mathew:2010ly}---that can be used to both promote and stigmatize behaviors through social forces such as peer pressure and adherence to norms. This has arguably been successful in achieving economically non-rational behavior on the Internet\footnote{Cases that come to mind include Stanford University returning their /8 block to IANA (``[A]s members of the network community, we need to think about this issue and do the right thing.... It's important for people that have large address space like ours to be good network neighbors.'', \url{http://www.networkworld.com/news/2000/0124ipv4.html}), aggregation due to the CIDR Report, responding to abuse, Conficker working group, etc.}, and perhaps even cases of collective action. While these social forces are probably less strong than they once were, and must now compete with considerations of Internet service as a commercial operation, they are still present as any participant in a network operator community can attest to. 

The purpose of this thesis is to investigate the effectiveness of one case of the use of social forces in the Internet operations community to solve an Internet collective action problem: the unsustainable growth of the Internet routing table. The objectives of this investigation are twofold. The first is to understand whether the monitoring mechanism---the CIDR Report---was effective in mitigating this problem by analyzing the behavior of networks that were deemed to be adversely affecting the scalability of the routing table and thus receive attention from the Internet operations community. The second is to draw conclusions about approaches to managing the Internet routing table (and possibly other collective action problems affecting Internet operations and infrastructure) based on the analysis of how this mechanism was or was not effective.
%
%These social forces may be useful and perhaps should be considered in designing solutions and building institutions to solve the challenges facing the Internet.

%%%%%%%%%%%%%%%%%%%%%%%%%%%%%%%%%%%%%%%%%%%%%%%%%%%%%%%%%%%%%%%%%%%%%%%%%%%%%%%%
\section{Context and Motivation}

While invisible to most Internet users, interdomain routing is one of the fundamental architectural elements of the Internet. Essentially the defining characteristic of the Internet as a network of networks, interdomain routing enables every network participating in the Internet to locate and exchange traffic with every other Internet-connected network.

In the face of continued growth, the interdomain routing system used by Internet networks faces scalability limits stemming from its design and operation that can limit the ability of network service providers to build and operate networks efficiently. The size and growth rate of the Internet routing table, and specifically the number of prefixes in the table, is one of the major sources of potential scaling issues \cite{rfc4984}. There are a number of causes for this growth, including the addition of new customers and networks in response to increased Internet demand and use, as well as network engineering and the expression of routing policy for existing networks. As shown in Figure \ref{fig:huston_table_plot}, the Internet routing table has grown at a super-linear, sometimes exponential, rate over time \cite{Huston:2001bs}.

\begin{figure}[h]
\begin{centering}
    \includegraphics[width=5in]{static_figures/huston_table_plot.png}
    \caption[Growth of the Internet (DFZ) routing table, 1994-present]{Growth of the Internet (DFZ) routing table, 1994-present \cite{6447-table-report}.}
    \label{fig:huston_table_plot}
\end{centering}
\end{figure}

% Some claim that engineering efforts -- Moore's law and etc -- make this a non-issue. others claim that the lack of feedback means that this issue lies in wait, and may come back to be problematic

% While currently not a problem, this issue came to the forefront once before in the early 90s. solution was technical change, coupled with social pressure to fix the problem, at a time when there is/was more pain

% we can model the problem as a common-pool resource problem. the internet is the 

Engineering efforts have allowed the capabilities of modern Internet routers to scale more quickly than the growth of the routing table for the most part, allowing the Internet to continue to grow organically without concern. Between evolutionary architectural improvements in succeeding generations of Internet routers \cite{McKeown:2006kx} and the near-guaranteed capacity and performance increases in semiconductors (i.e. Moore's Law), most routers today have capacities that exceed the needs of the Internet routing table. However, these engineering successes have not altered the underlying scaling properties of BGP, the current interdomain routing protocol. While some in the Internet engineering community claim that this is not a pressing concern \cite{Huston:2011ys, Huston:2009dq}, others \cite{Li:2011vn} claim that the lack of any mechanism to control or disincentivize routing table growth means that there is no guarantee that routing table growth will not outpace engineering developments in future.

Table growth causes engineering trade-offs to be made by router vendors \cite{Li:2011vn} and requires planning and investment consideration by network operators \cite{Zhao:2010fu} at present. While the actual size and growth rate of the routing table has not exceeded the advances provided by Moore's Law, it is exceeding Li's estimate of the constant cost sustainability as shown in Figure \ref{fig:li_router_scalability}. Further, the related challenges of IPv4 exhaustion (making more efficient use of existing IPv4 resources) and IPv6 adoption (growth of the IPv6 table on dual-stacked routers, which currently appears exponential\footnote{\url{http://bgp.potaroo.net/v6/as6447/}}) may cause routing table growth to accelerate towards unsustainability again. In the long term, unchecked growth could potentially curtail the decentralized, laissez-faire growth and operation that has been a hallmark of the Internet up to this point, or at least cause Internet routing to be more expensive than it might otherwise need to be.

% the theoretical\footnote{or even the more reasonable upper bound of $O(k\,2^{24})$ prefixes based on current RIR and operator policies} upper bound of $O(k\,2^{32})$ prefixes (where $k$ is a constant representing the number peers one connects to) is orders of magnitude beyond the capacity of any current router
\begin{figure}
\begin{centering}
    \includegraphics[width=5in]{static_figures/li_router_scalability.pdf}
    \caption[Normalized growth factors of the Internet routing table relative to Moore's Law]{Normalized growth factors of the Internet routing table relative to Moore's Law and other desirable engineering objectives, semiconductor chip cost and routing protocol convergence time, from 1999-2007 \cite{Li:2006cr}.}
    \label{fig:li_router_scalability}
\end{centering}
\end{figure}

The specter of routing table scaling problems has been encountered once before in the history of the Internet. In the early 1990s, as the Internet was becoming more commercial and moving from a single hierarchical backbone to multiple backbone providers, the Internet routing table began to grow at a rate that would exceed the capabilities of routers available at this time \cite{Huston:2001bs} and in some cases actually did affect the operational behavior of the routers \cite{Li:2011vn}. The solution to this problem was fundamentally a technical one: updating the addressing and routing architecture of the Internet to allow networks to be aggregated, or advertised in larger blocks, consuming fewer entries in the routing table. However, the adoption of these technical protocol changes and improvements in operational behavior required to enjoy benefit from these changes was promoted at least in part by social forces within the operator community. 

The mechanism that was used by the Internet operations community to promote more efficient route advertisements is called the CIDR Report. Transmitted weekly to the mailing lists associated with major network operator communities, it contains a section called the ``Aggregation Report'' that is an ordered list of the thirty networks that could most reduce the number of entries in the Internet routing table by improving their route announcement behavior. This email report, which first began in approximately 1994, was initially successful in affecting operator behavior. However, at some point it came to be viewed by operators as ineffective\footnote{From correspondence with Martin Hannigan, Patrick Gilmore, Tony Li, and Geoff Huston, as well as \cite{Steenbergen:2010nx}.}. It is interesting that this social mechanism, one that superficially seems to provide only very weak incentives and disincentives, was effective (or at least as claimed by a number of network operators and others in the community) in solving this collective action problem related to technology adoption and efficient route announcement.

%%%%%%%%%%%%%%%%%%%%%%%%%%%%%%%%%%%%%%%%%%%%%%%%%%%%%%%%%%%%%%%%%%%%%%%%%%%%%%%%
\section{The Case of the CIDR Report}

This thesis asks the question of whether the social forces in the Internet operations community are capable of inducing collective action. This question is asked in the context of one specifc case: the CIDR Report and its ability to control the growth of the Internet routing table. The working hypothesis for this question is that appearing on the CIDR Report did not significantly affect operator behavior. This position is based on discussions with network operators and observation of the continually-increasing number of prefixes in the routing table as shown in Figure \ref{fig:huston_table_plot}.

The case of the CIDR Report and routing table growth is of interest and particularly well suited for this analysis for a number of reasons. First, this case runs over a long period of time, starting in 1997, so there is a good opportunity to observe community and individual operator behavior. This long duration is backed by a large amount of publicly available data to support the analysis of the report, including the CIDR Report emails themselves as well as archived routing tables. The CIDR Report is naturally suited to allow for a quasi-experiment in that only part of the population is ``treated'' by the CIDR Report, while the rest of the population is available for use as a control. Finally, the CIDR Report was and is a well-known and well-publicized phenomenon within the operator community, and was created with the intent of educating and also socially pressuring operators, and so should be a suitable case for assessing the effects of social forces within the Internet operator community.

Other potential cases involving the social forces within the operator community, such as the backchannel communications mentioned in \cite{Mathew:2010ly}, are difficult to study as there is typically a lack of data, a lack of publicity of the events that occur that motivate social pressure, and the events of potential interest for analysis are somewhat ad-hoc and randomly distributed (i.e. the Youtube hijacking \cite{Brown:2008hc}).

As with any study of a single case, it is generally not possible to make generalizations about the broader question based the results of the study. Thus, while this thesis is motivated by the potential use of social forces to solve collective action problems of Internet operations, conclusions drawn from my study of the CIDR Report may not be useful in providing insights for other cases. However, any interesting insights about this case may be starting points for further study and exploration in other cases.

%%%%%%%%%%%%%%%%%%%%%%%%%%%%%%%%%%%%%%%%%%%%%%%%%%%%%%%%%%%%%%%%%%%%%%%%%%%%%%%%
\section{The Routing Table as a Common Pool Resource}

The unconstrained growth of the Internet routing table is often considered a commons problem: individuals derive private benefit from adding entries to the table, but each entry incurs a public cost---a negative externality---for all others that participate in the Internet routing system. The public cost is not necessarily trivial either, with one network operator roughly estimating the the marginal cost of a BGP prefix at \$6000-\$8000 per year \cite{Herrin:2008qa} and a researcher estimating the same figure at \$77,000 over the lifetime of the router \cite{Clayton:2010bh}. This cost is not borne by any one network, but is the estimated cost of the fraction of router resources consumed by one route across all BGP-speaking routers with a full (DFZ) routing table.

Discussions of the problem \cite{Huston:2001bs,Clayton:2010bh,Bellovin:2001qf} often invoke Hardin's \cite{Hardin:1968uq} notion of the tragedy of the commons---that individually rational actors will attempt to maximize consumption of a resource because they privately enjoy the full benefit of this consumption but share the cost of its reduced capacity, making all actors worse off. Common solutions advocated for such problems are the establishment of a central regulator or private property rights. However, the Internet routing table and interdomain routing lacks both such features and has not yet been reduced to tragedy.

In contrast to the broad notion of a ``commons problem'', Ostrom \cite{Ostrom:1990fv} presents the more nuanced concept of common pool resources (CPR) and CPR management problems. Her work has mostly focused on the management of natural resources such as fisheries, forests, and aquifers, but the general elements of the framework are applicable to other cases as well. In Ostrom's model of common pool resources, the \emph{resource system} (such as a fishery) is considered separately from the subtractable \emph{resource units} (in this example, fish) that can be extracted from the resource. The resource system produces some number of units that can be extracted sustainably, and beyond that point extraction causes harm to the system itself. Actors that extract resource units are referred to as \emph{appropriators} (fishers), and actors that take efforts to improve or sustain the resource (such as farming and stocking bodies of water with fish) are \emph{producers}. In all cases, the essential defining qualities of a CPR are that it is difficult to exclude others from using the resources (there are no private property rights), and that the resource is rivalrous (the use of the resource by one actor precludes its use by another).

The CPR framework can also be mapped relatively cleanly to the case of the Internet routing table. The Internet and it's interdomain routing system is not a natural resource, but it can still be viewed as a CPR resource system, with routing table capacity (``slots'') being the resource units. The value of the system is in its existence and continued functionality in interconnecting networks, rather than in the ``extraction'' of routing table slots, but 

-- arguably large service providers play a larger/more notable provider role

Mueller \cite{Mueller:2010bh} argues that IP addresses are indeed  and that actors establish regional internet registries (RIRs) as the CPR governance institution to govern both IP addresses and their allocation to by proxy manage the routing table.

interdomain routing as the common resource, routing table slots as the resource units

rational appropriators, various inputs
governance institutions affect the inputs

Communication is what prevents the tragedy from occurring.


the CIDR report was arguably a monitoring mechanism of whatever latent norm-based governance institution sort of existed, sort of offered to the community and taken up by some. not mutually agreed upon, but provides useful information that everyone who has a router can verify


and CPR governance institutions 


Hypothesis of method of action -- internalized norms and peer pressure -- Tony Li

``Proof'' in Huston's observation of routing table following IETF meetings \cite{Huston:2001bs}, \cite{Clayton:2010bh}.

social forces -- reputation and network standing \cite{Mathew:2010ly}

No explicit sanctions

If we accept the view of the routing table as a CPR and the CIDR report as part of the governance institution for the CPR, then we can use ostroms framework to look for variation that caused the report to change and perhaps for sources of inspiration for solving the problem.

%%%%%%%%%%%%%%%%%%%%%%%%%%%%%%%%%%%%%%%%%%%%%%%%%%%%%%%%%%%%%%%%%%%%%%%%%%%%%%%%
\section{Contributions}

This thesis makes a number of contributions to the space of Internet routing table analysis, including:
\begin{itemize}
\item{A history of the CIDR Report, as presented in Chapter \ref{chap:background}.}
\item{A well-documented, open-source implementation of the CIDR Report aggregation report algorithm that utilizes multiple vantage points, as described in Chapter \ref{chap:method} and Appendix \ref{chap:opensource}.}
\item{An analysis of the characteristics of the CIDR Report, including the distribution of networks that appear on it, etc., as presented in Chapter \ref{chap:analysis}.}
\item{An analysis of the effects of appearing on the CIDR Report on the route announcement behavior of individual networks, also presented in Chapter \ref{chap:analysis}.}
%\item{An analysis of network deaggregation behavior on a per-AS basis, unlike other table-wide analyses.}
\item{Consideration of the CIDR Report as CPR governance institution for the Internet Internet routing table.}
\end{itemize}

%%%%%%%%%%%%%%%%%%%%%%%%%%%%%%%%%%%%%%%%%%%%%%%%%%%%%%%%%%%%%%%%%%%%%%%%%%%%%%%%
\section{Roadmap}

The remainder of this thesis proceeds as follows. Chapter 2 presents a broad overview of relevant background information in this space, both for the reader who may be unfamiliar with interdomain routing on the Internet, as well as those who wish to understand some of the finer points that motivate Internet operations and routing table growth. This chapter also contains an overview CIDR Report and a brief history of the events that motivated its creation and evolution with time. Chapter 3 describes the analytical approach taken to determine whether the CIDR report was effective, as well as detailing the data sources and algorithms that were used to implement the analysis and generate results. Chapter 4 presents and discusses the results of this analysis: it considers both the overall characteristics of the CIDR report and the networks that appear on it, as well as a specific analysis of the behavior of networks that appear on the CIDR Report. Following the presentation of these analytical results, Chapter 5 discusses a number of reasons for why the efficacy of the CIDR Report may have changed over time by considering the situation through the lens of Ostrom's common pool resource framework. Chapter 6 presents a review of related work and literature. Finally, Chapter 7 offer a concluding discussion and recommendations regarding the management of the Internet routing table, as well as how these observations might be used to design institutions to solve other problems facing the Internet.