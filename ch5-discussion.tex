\chapter{\fbox{TODO:Discussion}}
\label{chap:discussion}

% THE ENTIRE AND SOLE POINT OF THE DISCUSSION SECTION SHOULD BE TO INTERPRET THE RESULTS AS TO WHETHER THE CIDR REPORT EVER WAS EFFECTIVE AND THEN, BASED ON THE OBSERVED CHANGES, WHETHER IT BECAME LESS EFFECTIVE OR WHETHER SOMETHING ELSE CAUSED THE OBSERVED BEHAVIORAL CHANGE

The challenge of agreeing to supply an institution that does anything more than provide information (i.e. sanctions) multilaterally is potentially problematic in looking beyond the CIDR Report.

\fbox{Look at Tony Li's interview again}

\section{Was the CIDR Report effective?}
- counterfactual plots
- in analysis/counterfactual

\section{Was the CIDR Report accurate?}
\subsection{The importance of vantage point selection}
- the importance of vantage points

From Patrick Gilmore:

\begin{quote}
P.S. The list is not 100\% correct.  For instance, tw telecom has no-export set on their de-agg prefixes.  This means those de-aggs don't hit the global table, just their transit provider.  But if Geoff peers with the transit provider directly, he may see them (even though he should not).  I have spoken to tw telecom about this face-to-face.  (Part of that "shaming" thing.)
\end{quote}

\subsection{Is all deaggregation equal?}
Yes, all deaggregation affects the routing table in the same way, but the costs and benefits system-wide are different.
- ``good'' vs. ``bad'' deaggregation.

\section{What caused the observed behavior changes?}
\subsection{Changing community norms and responses}
\subsection{Changing benefits of routing deaggregation}
\subsection{Changing costs of routing deaggregation and appearing on the CIDR Report}


% OR conversely
% 1) quality of signal
% 2) response to signal
