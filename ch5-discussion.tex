\chapter{\fbox{TODO:Discussion}}
\label{chap:discussion}

% THE ENTIRE AND SOLE POINT OF THE DISCUSSION SECTION SHOULD BE TO INTERPRET THE RESULTS AS TO WHETHER THE CIDR REPORT EVER WAS EFFECTIVE AND THEN, BASED ON THE OBSERVED CHANGES, WHETHER IT BECAME LESS EFFECTIVE OR WHETHER SOMETHING ELSE CAUSED THE OBSERVED BEHAVIORAL CHANGE

The challenge of agreeing to supply an institution that does anything more than provide information (i.e. sanctions) multilaterally is potentially problematic in looking beyond the CIDR Report.

\fbox{Look at Tony Li's interview again}

\section{Was the CIDR Report accurate?}
\subsection{The importance of vantage point selection}
- the importance of vantage points

From Patrick Gilmore:

\begin{quote}
P.S. The list is not 100\% correct.  For instance, tw telecom has
no-export set on their de-agg prefixes.  This means those de-aggs don't hit the
global table, just their transit provider.  But if Geoff peers with the transit
provider directly, he may see them (even though he should not).  I have spoken
to tw telecom about this face-to-face.  (Part of that "shaming" thing.)
\end{quote}

\subsection{Is all deaggregation equal?}
Yes, all deaggregation affects the routing table in the same way, but the costs
and benefits system-wide are different.
- ``good'' vs. ``bad'' deaggregation.

\section{Was the CIDR Report effective?}
- counterfactual plots
- in analysis/counterfactual

As those that have been treated by the CIDR Report have become more outliers
over time, it's possible that those below the outliers may have been more
susceptible to treatmetn. Or, they may have moved on tehir own -- it would be
interesting to sample those with netgain comparable to 1998 CIDR report from
2011 and see if the behavior is more similar to 1998 or 2011.

\subsection{Interpreting the plots}

How quickly must the CIDR Report have had to been effective for one to consider
it having "worked" as opposed to measuring some other phenomenon?

In spite of seeing aggregation behavior relative to the control, the total
number of routes in the routing table does not drop over time, suggesting that
old people are replaced by new ``bad guys'' or new entrants. Cittadini's study
suggests that the proportion of ``bad guys'' remains the same over time (the
people that would show up on the CIDR Report), suggesting the rolling-over bad
guy theory.

OR. The table just grew less quickly than it otherwise could have, but we can't
examine the counterfactual world...?

Rolling-over bad guy theory would be supported by the rate of growth of new
ASes on teh CIDR Report?

Arthur suggests looking at the fraction (\# of /24s / \# of prefixes in the
routing table) per AS or for the entire routing table, over time

What do Cittadini's observations mean for my analysis of the later repport
where people are far more deaggregated

\section{What caused the observed behavior changes?}
\subsection{Changing community norms and responses}
\subsection{Changing benefits of routing deaggregation}
\subsection{Changing costs of routing deaggregation and
appearing on the CIDR Report}


% OR conversely 1) quality of signal 2) response to signal
